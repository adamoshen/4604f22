\documentclass[11pt,letterpaper]{article}
\usepackage[margin=0.7in]{geometry}
\usepackage{amsmath}
\usepackage{amsfonts}
\usepackage{amssymb}
\usepackage{parskip} % For proper spacing
\setlength{\parindent}{0em}
\setlength{\parskip}{3mm}
%\pagenumbering{gobble}
%\usepackage{graphicx}
%\graphicspath{}
%\usepackage[hidelinks]{hyperref}
\usepackage{color}
\usepackage[dvipsnames]{xcolor} % For additional colours
\usepackage{enumitem} % Enumerate labels
\allowdisplaybreaks
\title{Lab 1 Notes}
\author{Adam Shen}
\date{}

\newcommand{\E}[1]{\mathbf{E}\left(#1\right)}
\newcommand{\Var}[1]{\mathbf{Var}\left(#1\right)}
\newcommand{\Prob}[1]{\mathbf{P}\left(#1\right)}
%\newcommand{\nh}{H_{0}:\,}
%\newcommand{\ah}{H_{A}:\,}
\newcommand{\blue}[1]{{\color{blue}#1}}
\newcommand{\red}[1]{{\color{red}#1}}
\definecolor{green}{HTML}{009900}
\newcommand{\green}[1]{{\color{green}#1}}
\definecolor{lorange}{HTML}{E69F00}
\definecolor{lblue}{HTML}{56B4E9}
\definecolor{teal}{HTML}{009E73}
\definecolor{yellow}{HTML}{F0E442}
\definecolor{dblue}{HTML}{0072B2}
\definecolor{dorange}{HTML}{D55E00}
\definecolor{pink}{HTML}{CC79A7}

\begin{document}
\maketitle

The given Pareto distribution is

\[f(x) \,=\, \frac{\alpha 3^{\alpha}}{x^{\alpha+1}}, \quad x > 3, \quad \alpha > 1,\]

and zero otherwise.

\begin{enumerate}[label=(\roman*)]
\item Find the cdf.
\blue{
\begin{align*}
F(x) \,&=\, \int\limits_{-\infty}^{x}\frac{\alpha 3^{\alpha}}{x^{\alpha+1}}\,dt\\[3mm]
\,&=\, \int\limits_{3}^{x}\alpha\,3^{\alpha}\,t^{-(\alpha + 1)}\,dt\\[3mm]
\,&=\, \left. \frac{\alpha\,3^{\alpha}\,t^{-(\alpha + 1) + 1}}{-(\alpha + 1) + 1} \right|_{t=3}^{t=x}\\[3mm]
\,&=\, \left. -3^{\alpha}\,t^{-\alpha}\right|_{t=3}^{t=x}\\[3mm]
\,&=\, -3^{\alpha}(x^{-\alpha} \,-\, 3^{-\alpha})\\[3mm]
\,&=\, -\frac{3^{\alpha}}{x^{\alpha}} \,+\, 1\\[3mm]
\,&=\, 1 \,-\, \left(\frac{3}{x}\right)^{\alpha}, \quad x > 3
\end{align*}
and $F(x) = 0$ for $x < 3$.
}

\item Find the quantile function.

Recall that the cdf was defined as

\[F(x) \,=\, \Prob{X \leq x} \,=\, p,\]

where $x$ is the \emph{quantile} and is given, and $p$ is the probability that we need to often
need to calculate. With the quantile function, the conditions are reversed:

\[Q(p) \,=\, x \quad\text{such that}\quad \Prob{X \leq x} \,=\, p,\]

where $p$ is given is $x$ is the value we seek. If $F(x)$ is a one-to-one function, then $Q(p)$ is
simply the inverse of $F(x)$.

\blue{
Using cdf from (i), we solve for $x$:
\begin{align*}
p \,&=\, 1 \,-\, \left(\frac{3}{x}\right)^{\alpha}\\[3mm]
\left(\frac{3}{x}\right)^{\alpha} \,&=\, 1-p\\[3mm]
\frac{3}{x} \,&=\, (1-p)^{1/\alpha}\\[3mm]
x \,&=\, \frac{3}{(1-p)^{1/\alpha}}
\end{align*}
Therefore, our quantile function is:

\[Q(p) \,=\, \frac{3}{(1-p)^{1/\alpha}}, \quad 0 \leq p \leq 1\]
}
\end{enumerate}

\end{document}